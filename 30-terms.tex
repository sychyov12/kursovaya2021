\chapter{Термины, определения и обозначения}
\label{cha:ch_1}
\newcounter{MYc} 
\def\MYhyp{\addtocounter{MYc}{1}{\bf Определение \arabic{MYc}: }}
\par
В настоящей дипломной работе применяются следующие термины и обозначения с соответствующими определениями.
\par
\textbf{Дополнение} "--- приписывание дополнительных бит к строке бит.
\par
\textbf{Инициализационный вектор} "--- вектор, определенный как начальная точка работы функции хэширования.
\par
\textbf{Ключ подписи} "--- элемент секретных данных, специфичный для субъекта и используемый только данным субъектом в процессе формирования цифровой подписи.
\par
\textbf{Ключ проверки подписи} "--- элемент данных, математически связанный с ключом подписи и используемый проверяющей стороной в процессе проверки цифровой подписи.
\par
\textbf{Параметр схемы ЭЦП} "--- элемент данных, общий для всех субъектов схемы цифровой подписи, известный или доступный всем этим субъектам.
\par
\textbf{Подписанное сообщение} "--- набор элементов данных, состоящий из сообщения и дополнения, являющегося частью сообщения.
\par
\textbf{Последовательность псевдослучайных чисел} "--- последовательность чисел, полученная в результате выполнения некоторого арифметического (вычислительного) процесса, используемая в конкретном случае вместо последовательности случайных чисел.
\par
\textbf{Последовательность случайных чисел} "--- последовательность чисел, каждое из которых не может быть предсказано (вычислено) только на основе знания предшествующих ему чисел данной последовательности.
\par
\textbf{Процесс проверки подписи} "--- процесс, в качестве исходных данных которого используются подписанное сообщение, ключ проверки подписи и параметры схемы ЭЦП, результатом которого является заключение о правильности или ошибочности цифровой подписи.
\par
\textbf{Процесс формирования подписи} "--- процесс, в качестве исходных данных которого используются сообщение, ключ подписи и параметры схемы ЭЦП, а в результате формируется цифровая подпись.
\par
\textbf{Свидетельство} "--- элемент данных, представляющий соответствующее доказательство достоверности (недостоверности) подписи проверяющей стороне.
\par
\textbf{Случайное число} "--- число, выбранное из определенного набора чисел таким образом, что каждое число из данного набора может быть выбрано с одинаковой вероятностью.
\par
\textbf{Сообщение} "--- строка бит произвольной конечной длины.
\par
\textbf{Функция сжатия} "--- итеративно используемая функция, преобразующая строку бит длиной $L_1$ и полученную на предыдущем шаге строку бит длиной $L_2$ в строку бит длиной $L_2$.
\par
\textbf{Хэш-код} "--- строка бит, являющаяся выходным результатом хэш-функции.
\par
\textbf{Хэш-функция} "--- функция, отображающая строки бит в строки бит фиксированной длины и удовлетворяющая следующим свойствам:
\begin{enumerate}
	\item по данному значению функции сложно вычислить исходные данные, отображаемые в это значение;
	\item для заданных исходных данных сложно вычислить другие исходные данные, отображаемые в то же значение функции;
	\item сложно вычислить какую-либо пару исходных данных, отображаемых в одно и то же значение.
\end{enumerate}
\par
\textbf{Электронная цифровая подпись (ЭЦП)} "--- строка бит, полученная в результате процесса формирования подписи.
\par
$V^*$ "--- множество всех двоичных векторов-строк конечной размерности (далее - векторы), включая пустую строку.
\par
$|A|$"--- размерность (число компонент) вектора $A\in V^*$ (если $A$ - пустая строка, то $|A|=0$).
\par
$V_n$ "--- множество всех $n$-мерных двоичных векторов, где $n$ - целое неотрицательное число; нумерация подвекторов и компонент вектора осуществляется справа налево, начиная с нуля.
\par
$\oplus$ "--- операция покомпонентного сложения по модулю 2 двух двоичных векторов одинаковой размерности.
\par
$A\|B$ "--- конкатенация векторов $A,B\in V^*$, т.е. вектор из $V_{|A|+|B|}$, в котором левый подвектор из $V_{|A|}$ совпадает с вектором $А$, а правый подвектор из $V_{|B|}$ совпадает с вектором $В$.
\par
$A^n$ "--- конкатенация $n$  экземпляров вектора $A$.
\par
$\mathbb{Z}_{2^n}$ "--- кольцо вычетов по модулю $2^n$.
\par
$\boxplus$ "--- операция сложения в кольце $\mathbb{Z}_{2^n}$.
\par
$Vec_n:\mathbb{Z}_{2^n} \to V_n$ "--- биективное отображение, сопоставляющее элементу кольца $\mathbb{Z}_{2^n}$ его двоичное представление, т.е. для любого элемента $z \in \mathbb{Z}_{2^n}$ представленного вычетом $z_0+2z_1 + \dots 2^{n-1}z_{n-1}$, где $z_j \in \{0,1\},\,j=0,\dots,n-1$, выполнено равенство $Vec_n(z)=z_{n-1}\|\dots\|z_1\|z_0$.
\par
$Int_n\colon V_n \to \mathbb{Z}_{2^n}$ "--- отображение, обратное отображению $Vec_n$, т.е. \\ $Int_n = Vec_n^{-1}$.
\par
$MSB_n\colon V^* \to V_n$ "--- отображение, ставящее в соответствие вектору \\ $z_{k-1}\|\dots\|z_1\|z_0,\:k\geqslant n$, вектор $z_{k-1}\|\dots\|z_{k-n+1}\|z_{k-n}$.
\par
$a:=b$ "--- операция присваивания переменной $a$ значения $b$.
\par
$\Phi\circ\Psi$ "--- произведение отображений, при котором отображение $\Psi$ действует первым.
\par
$M$ "--- двоичный вектор, подлежащий хэшированию, $M \in V^*,\; |M|<2^{512}$.
\par
$H \colon V^* \to V_n$ "--- функция хэширования, отображающая вектор (сообщение) $M$ в вектор (хэш-код) $H(M)$.
\par
$IV$ "--- инициализационный вектор функции хэширования, $IV \in V_{512}$.
\par
$Z$ "--- множество всех целых цисел.
\par
$p$ "--- простое число, $p>3$.
\par
$F_p$ "--- конечное простое поле, представленное как множество  из $p$ наименьших неотрицательных вычетов $\{0,\;1,\;\dots,\;p-1\}$.
\par
$b\;(mod\; p)$ "--- минимальное неотрицательное число, сравнимое с $b$ по модулю $p$.
%\par
%$a, b$ "--- коэффициенты эллиптической кривой;
%\par
%$m$ "--- порядок группы точек эллиптической кривой;
%\par
%$q$ "--- порядок подгруппы группы точек эллиптической кривой;
%\par
%$O$ "--- нулевая точка эллиптической кривой;
%\par
%$P$ "--- точка эллиптической кривой порядка $q$;
%\par 
%$d$ "--- целое число "--- ключ подписи;
%\par
%$Q$ "--- точка эллиптической кривой "--- ключ проверки подписи;
%\par
%$\zeta$ "--- цифровая подпись над сообщением $M$;

