\chapter{Первая глава. Описание алгоритмов}
\label{cha:ch_1}
\section{Игры в развернутой форме и равновесие Нэша}
\par
Игра в развернутой форме представляют компактную общую модель взаимодействий между агентами и явно отражает последовательный характер этих взаимодействий. Последовательность принятия решений игроками в такой постановке представлена деревом решения. При этом листья дерева отождествлены с терминальными состояния, в которых игра завершается и игроки получают выплаты. Любой нетерминальный узел дерева представляет точку принятия решения. Неполнота информации выражается в том, что различные узлы игрового дерева считаются неразличимыми для игрока. Совокупность всех попарно неразличимых состояний игры называется информационными наборами. Приведем формальное определение.
\par
Определение 1: Конечная игра в развернутой форме с неполной информацией содержит следующие компоненты:
\begin{itemize}
	\item Конечное множество игроков $N$;
	\item Конечное множество историй действий игроков $H$, такое, что $\emptyset \in H$ и любой префикс элемента из $H$ также принадлежит $H$. $Z \subseteq H$ представляет множество терминальных историй (множество историй игры на являющихся префиксом). $A(h)=\{a \colon (h,\;a)\in H \}$ "--- доступные после нетерминальной истории $h\in H$ действия;
	\item Функция $P\colon H\setminus Z \to N \cup \{c\}$, которая сопоставляет каждой нетерминальной истории $h\in H\setminus Z$ игрока, которому предстоит принять решение, либо игрока $c$ представляющего случайное событие;
	\item Функция $f_c$, которая сопоставляет всем $h \in H$, для которых $P(h)=c$, вероятностное распределение $f_c(\cdot |h)$ на $A(h)$. $f_c(a|h)$ представляет вероятность выбора $a$ после истории $h$;
	\item Для каждого игрока $i \in N$ $\mathcal{I}_i $ обозначает разбиение $ \{h \in H \colon P(h) = i\}$, для которого $A(h)=A(h')$ всякий раз когда $h$ и $h'$ принадлежат одному члену разбиения. Для $I_i \in \mathcal{I}_i$ определим $A(I_i)=A(h)$ и $P(I_i)=i$ для всех $h \in I_i$. $\mathcal{I}_i$ называют информационным разбиением игрока $i$, а $I_i \in \mathcal{I}_i$ информационным набором игрока $i$;
	\item Для каждого игрока $i \in N$ определена функция выигрыша $u_i \colon Z \to R$. Если для игры в развернутой форме выполняется $\forall z \in Z \sum_{i \in N}U_i(z) = 0 $, то такую игру называют игрой с нулевой суммой. Определим $\Delta_{u,i} = max_{z\in Z}\;u_i(z) - min_{z\in Z}\;u_i(z)$ для диапазона выплат игрока. 
\end{itemize}
\par
 Отметим, что информационные наборы могут использоваться не только для реализации правил конкретной игры, но и могут быть использованы для того, чтобы заставить игрока забыть о предыдущих действиях. Игры в которых игроки не забывают о действиях называют играми с полной памятью. В дальнейшем мы будем рассматривать конечные игры в развернутой форме с нулевой суммой и полной памятью.

Стратегия игрока $i$ "--- это функция $\sigma_i$, которая ставит в соответствие каждому информационному набору $I_i \in \mathcal{I}_i$ вероятностное распределение на $A(I_i)$. Обозначим за $\Sigma_i$ множество всех стратегий игрока $i$. Стратегический профиль $\sigma$ содержит стратегии для каждого игрока $i \in N$. При этом за $\sigma_{-i}$ обозначим $\sigma$ без $\sigma_i$. 

Обозначим за хъ вероятность того, что игроки достигнут хъ руководствуясь хъ. Мы можем представить хъ как хъ, выделяя вклад каждого игрока. В таком случае, хъ обозначает вероятность принятия совокупности решений, ведущих от префикса хъ к префиксу хъ для участков в которых хъ. Аналогично обозначим за хъ вероятность достижения истории хъ всеми игроками за исключением хъ. Формально можно определить предыдущие величины следующим образом 

Хъ 

Для хъ определим хъ. Аналогично введем хъ и хъ. 

Ожидаемое среднее значение выплат для игрока хъ обозначим как хъ. 

Традиционным способом решения игр в развернутой форме является поиск равновесного профиля стратегий хъ, который удовлетворяет следующему условию. 

Хъ 

Такой стратегический профиль называют равновесием по Нэшу. 

В случае, если некий стратегический профили хъ удовлетворяет условию 

Хъ 

Его называют хъ – равновесием 



Контрафактические сожаления и их минимизация 

Минимизация сожалений является популярным концептом, для построения итеративных алгоритмов приближенного решения игр в развернутой форме. Приведем связанные с ней определения. Рассмотрим дискретный отрезок времени T включающий 1, T раундов. Обозначим за хъ стратегию игрока хъ в раунде хъ. 

Определение. Средним общим сожалением игрока хъ на момент времени T называют величину 

Хъ 

В дополнении к этому определим хъ как среднюю стратегию относительно всех раундов от 1 до T. Таким образом для каждого хъ и хъ введем 

Хъ 

Теорема. Если на момент времени T средние общие сожаления игроков меньше хъ, то хъ является 2хъ равновесием. 

Говорят, что алгоритм выбора хъ реализует минимизацию сожалений, если средние общие сожаления игроков стремятся к нулю при t стремящимся к бесконечности. И как результат, алгоритм минимизации сожалений может быть использован для нахождения приближенного равновесия по Нэшу. 

Понятие контрафактического сожаления служит для декомпозиции среднего общего сожаления в набор дополнительных сожалений, которые могут быть минимизированы независимо для каждого информационного набора.  

Обозначим через хъ цену игры с точки зрения истории хъ, при условии, что хъ была достигнута, и игроки спользуют в дальнейшем хъ. 

Определение. Контрафактической ценой хъ назовем ожидаемую цену при условии, что информационный набор хъ был достигнут, когда все игроки кроме хъ играли в соответствии с хъ. Формально 

Хъ, 

где хъ вероятность перехода из хъ в хъ. 

Обозначим за хъ стратегический профиль идентичный хъ за исключением того, что хъ всегда выбирает хъ попадая в хъ. 

Немедленным контрафактическим сожалением назовем 

Хъ 

Интуитивно это выражение можно понимать, как аналог среднего общего сожаления в терминах контрафактической цены. Однако вместо рассмотрения всевозможных максимизирующих стратегий рассматриваются локальные модификации стратегии. Хъ. Связь немедленных контрафактических сожалений и общих средних сожалений раскрывает следующая теорема. 

Теорема. Хъ 

Таким образом, минимизация немедленных контрафактических сожалений минимизирует общие сожаления. В свою очередь минимизация немедленного контрафактического сожаления может происходить за счет минимизации выражений под функцией максимума. Таким образом мы приходим к понятию контрафактического сожаления 

Хъ. 

Контрафактическое сожаление рассматривает действие в информационном наборе. В свою очередь для минимизации контрафактических сожалений можно применить алгоритм приближения Блэквела, который применимо к рассматриваемым сожалениям приведет к следующей последовательности стратегий 

Хъ 

Другими словами, действие выбирается в пропорции соотношения позитивных контрафактических сожалений для не выбора этого действия. Обоснование сходимости полученного решения и оценку ее скорости предоставляет следующая теорема. 

Теорема. Если игроки придерживаются стратегий, заданных выражением (), то хъ 
\section{Теорема Блэквела и минимизация сожалений}
\section{Контерфактические сожаления и их минимизация}
\section{Метод Монте-Карло}