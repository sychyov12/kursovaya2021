\Introduction

\par
Данная работа посвящена вопросам обучения основам криптографии, в частности аффинному шифрованию. Данный вид шифрования является базовым и достаточно прост, чтобы произвести процедуру шифрования на листе бумаги, но при этом требует от учащегося знаний в области арифметики в кольце вычетов, а также принципов формирования входных и выводимых данных алгоритма шифрования.
\par
Данная тема рассматривается в курсах криптографии, как один из классических примеров, а также, в виду относительной простоты, может быть интересна всем желающим попрактиковаться в шифровании сообщений.
\par
Актуальность поставленной задачи связана прежде всего с возможностью использования программы для проверки навыков шифрования у учащихся. Помимо всего прочего, применение автоматизированной программы на компьютере может помочь не только проверить результат, но и сформировать корректную задачу, что может быть полезным.
\par
Объект исследования данной работы - обучение основам криптографии.
\par
Предмет исследования – системы тестирования навыков шифрования.
\par
Ставится задача: разработать компьютерную программу способную за небольшой промежуток времени(от 5 до 30 минут в зависимости от сложности задания) оценить степень подготовки ученика в вопросах основ криптографии путем тестирования.
\par
В данной работе выделены следующие подзадачи:
\begin{itemize}
	\item подготовка алгоритмов;
	\item реализация пользовательского интерфейса.
\end{itemize}
\par
Подготовка алгоритмов включает постановку задачи и создание всех специфичных классов, отвечающих за формирование, хранение и обработку необходимых данных.
\par
Реализация пользовательского интерфейса включает создание визуальной оболочки программы ее непосредственную реализацию на выбранной платформе.
