\Introduction

\par
В современном мире защита информации является неотъемлемой частью жизни общества. Криптография бывает необходима в самых разных сферах будь то переписка или доступ к документации. Особую важность имеет надежность и достоверность обмена информацией между государственными структурами, ведь любая невыявленная подделка может привести к колоссальным убыткам и нарушению общественного порядка. С другой стороны, проволочки в документообороте могут мешать оперативному реагированию, что недопустимо при наличии лучших альтернатив. Таким образом, среди прочего, возникает необходимость на высшем государственном уровне определить актуальные процедуры, позволяющие оперативно определить целостность сообщения и достоверность отправителя.
\par
Ключевую роль в воросах удостоверения информации играет электронная цифровая подпись, которая позволяет установить не только подлинность отправителя но и целостность сообщения. Как правило, в процедурах генерации и проверки цифровой подписи используется хэш-код подписываемого сообщения.
\par
Целью данной работы является программная реализация функции хэширования и цифровой подписи согласно ГОСТ Р 34.11-2012 и ГОСТ Р 34.10-2012 соответственно. Все специфичные классы предполагается реализовать на языке программирования C\# в составе среды Visual Studio 2019. Выбор обусловлен популярностью данной среды разработки. В итоге, планируется реализовать хэш-функцию , процедуру генерации цифровой подписи и процедуру проверки цифровой подписи.
\par
Объект исследования данной работы"--- криптографическая защита информации.
\par
Предмет исследования"--- функция хэширования, процессы формирования и проверки электронной цифровой подписи.
\par
В рамках реализации намеченой цели можно выделить следующие подзадачи:
\begin{itemize}
	\item анализ ГОСТ Р 34.11-2012;
	\item реализация функции хэширования согласно ГОСТ Р 34.11-2012;
	\item анализ ГОСТ Р 34.10-2012;
	\item реализация цифровой подписи согласно ГОСТ Р 34.10-2012;
	\item анализ результатов.
\end{itemize}
\par
Данная тема может быть интересна тем, кто заинтересован в реализации функции хэширования и цифровой подписи в различных программных средах. Данная работа также раскрывает некоторые прикладные вопросы процесса реализации. Актуальность выбранной темы обусловлена перспективой внедрения новых стандартов в государственные информационные системы.
