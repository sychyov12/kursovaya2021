\Introduction

\par
В настоящее время под явлением коррупции понимается неправомерное использование должностных привилегий в личных целях. Например, к коррупции относят присвоение ренты и получение взяток. При этом, некоторые из этих процессов могут быть описаны с помощью моделей теории игр. Одной из таких моделей является модель совместного преступления\cite{Spengler}. Такой подход подразумевает эндогенную природу игровых ситуаций и ассиметричные выплаты. Похожий вопрос, связанный с механизмом повторной проверки, в достаточно большом объеме изучен для однородного набора из нескольких проверяемых лиц\cite{Kumacheva}. В то же время, проверяемые лица могут иметь свою организационную структуру, наличие которой может вносить значительные коррективы в распределение информации на различных этапах игры. Например, за счет иерархии можно реализовать механизм дополнительных проверок\cite{Orlov}. Имеет смысл построение и анализ более обобщенных с точки зрения организационной структуры и распределения информации моделей.
\par
Подобные модели могут быть представлены как игры в развернутой форме. При такой постановке задачи можно близким к естественному способом отразить в игровой форме структуру последовательного принятия решений набором участников в конфликтной ситуации.
\par
Существуют методы позволяющие достаточно эффективно сформировать приближенное кореллированное равновесие для заданной игры в развернутой форме с неполной информацией. Довольно популярен итеративный алгоритм минимизации контрафактических сожалений (Conterfactual Regret Minimization)\cite{NIPS07cfr} и его модификация предусматривающая использование метода Монте-Карло (MCCFR)\cite{MCCFR}.  Данные алгоритмы появились не так давно, но на их основе уже получен ряд недостижимых до этого по сложности результатов.
\par
Целью данной работы является изучение возможности применения алгоритма CFR в задачах моделирования корррупции и его отработка на практике для модели раскрытия совместного преступления и для модели коррупции в иерархической структуре.
\par
В соответствии с темой работы были поставлены следующие задачи:
\begin{itemize}
	\item выбор конкретной формы модели с использованием уже существующихи улучшил предыдущие подобные модели;
	\item выбор алгоритма для поиска решения и анализа модели;
	\item проведение расчетов и анализ результатов.
\end{itemize}
\par
Данная работа может быть интересна людям желающим ознакомится с некоторыми современными техниками решения игр с неполной информацией.
