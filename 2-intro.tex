\Introduction

\par
В последнее время математическая теория игр с неполной информацией находит все большее применение в таких отраслях как теория операций, экономика, кибербезопасность и физическая безопасность. Не в последнюю очередь это происходит благодаря постоянному совершенствованию алгоритмов и увеличению производительности вычислительной техники. Частным случаем таких игр являются игры в развернутой форме. При такой постановке задачи можно близким к естественному способом отразить в игровой форме структуру последовательного принятия решений набором участников в конфликтной ситуации.
\par
Существенным шагом в развитии данного направления является использование алгоритма подсчета сожалений (regret matching). Алгоритм предполагает итеративное вычисление последовательности стратегий в среднем сходящейся к оптимальному стратегическому профилю. Открытие этого метода привело к появлению ряда алгоритмов для поиска приближенного решения в играх с неполной информацией.
\par
Данная работа посвящена рассмотрению одного из популярных в настоящее время итеративных алгоритмов -  контерфактической минимизации сожалений (Conterfactual Regret Minimization) и его модификации приедусматривающей использование метода монте карло (MCCFR). Данные алгоритмы появились не так давно, но на их основе уже получен ряд недостижимых до этого по сложности результатов. 
\par
Целью данной работы является описание и отработка на практике приведенных выше алгоритмов.
\par
В соответствии с темой работы поставлены следующие задачи:
\begin{itemize}
	\item подготовить теоритическое описание алгоритмов;
	\item выделить некоторые игры в развернутой форме для последующего решения;
	\item реализовать алгоритм и произвести расчет стратегического профиля для приведенных игр.
\end{itemize}
\par
Данная работа может быть интересна людям желающим ознакомится с некоторыми современными техниками решения игр с неполной информацией.
