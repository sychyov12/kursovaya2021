\begin{center}
	{\large \textbf{Реферат}}
\end{center}
\par
Дипломная работа <<Программная реализация современных отечественных стандартов функции хэширования и цифровой подписи>> 72 стр., 11 илл., 14 источников, 4 прил.
\par
Ключевые слова: информационные технологии, защита информации, функция хэширования, цифровая подпись, эллиптическая кривая, длинная арифметика, ГОСТ Р 34.11-2012, ГОСТ Р 34.10-2012.
\par
объектом исследования являются функция хэширования и цифровая подпись.
\par
Целью работы является программная реализация функции хэширования и цифровой подписи согласно ГОСТ Р 34.11-2012 и ГОСТ Р 34.10-2012 соответственно. В работе, для реализации процессов вычисления функции хэширования, применяются методы битовой арифметики и объектно ориентированного программирования на примере языка программирования C\#. Для функции хэширования рассматриваются ряд оптимизаций, основанных на группировке операций и использовании таблицы предпросчета. Процессы формирования и проверки цифровой подписи основаны на использовании готовой библиотеки для работы с большими числами и дополнены сереализуемыми параметрами.
\par Полученные в результате работы программные классы могут использоваться для практического расчета функции хэширования и процессов формирования и проверки цифровой подписи. Дополнительно реализован ряд приложений непосредственно их использующих.