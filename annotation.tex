{\Large Программная реализация современных отечественных стандартов функции хэширования и цифровой подписи
\par}
{\large 
	Сычев Роман Сергеевич, математика  и компьютерные науки (кафедра алгебры и математической логики)\\
	Научный руководитель: Яблокова С. И., кандидат ф.-м. н., доцент
\par}
Данная работа рассматривает вопросы реализации практического вычисления функции хэширования и цифровой подписи согласно ГОСТ Р 34.11-2012 и ГОСТ Р 34.10-2012. В работе приводится описание используемых в стандартах алгоритмов и рассматриваются некоторые возможные программные реализации с применением методов объектно ориентированного программирования.
\par
Работа включает четыре главы. 
\par
В первой главе приведено описание функции хэширования согласно ГОСТ Р 34.11-2012. Стандарт ГОСТ Р 34.11-2012 определяет семейство хэш-функций состоящее из двух функций с длинами хэш-кода равными 256 и 512 битам соответственно. Описание включает параметры алгоритма, основные преобразования, функцию сжатия и алгоритм вычисления. Среди преобразований выделены: перестановка байт, подстановка бит, побитовое сложение и умножение на матрицу справа. Перечисленные преобразования задаются стандартом в явном виде и используются в функции сжатия. Функция сжатия в процессе вычисления проходит 12 итераций, на каждой из которых используется специальный ключ. Значения раундовых ключей также определены в параметрах.
\par
Во второй главе рассмотрена программная реализация функции хэширования. Для функции хэширования рассматриваются ряд оптимизаций, основанных на группировке операций и использовании таблицы предпросчета. Рассматриваются две основные схемы вычисления. Первая схема предполагает расчет хэша от массива байт. Вторая схема основана на поточной реализации, которая не требует чтения источника целиком. Все необходимые методы определены в отдельном программном классе, который можно использовать в составе библиотек.
\par
В третьей главе приведено описание процессов формирования и проверки цифровой подписи согласно ГОСТ Р 34.10-2012. Описание включает группу точек эллиптической кривой, параметры схемы подписи, процесс формирования подписи и процесс проверки подписи. Коэффициенты эллиптической кривой задаются в явном виде или с помощью инварианта эллиптической кривой. В процессах формирования и проверки подписи используется функция хэширования, определенная в ГОСТ Р 34.11-2012. По аналогии с функцией хэширования, схемы цифровой подписи предполагают два варианта работы в зависимости от длины хэша.
\par
Четвертая глава посвящена  вопросам программной реализации цифровой подписи. Процессы формирования и проверки цифровой подписи основаны на использовании готовой библиотеки для работы с большими числами и дополнены сереализуемыми параметрами. В ходе расчетов используется расширенный алгоритм Евклида и алгоритм быстрого скалярного умножения. В отдельные типы выделены параметры схемы цифровой подписи, ключ подписи, ключ проверки подписи и цифровая подпись. Все необходимые методы определены в отдельном программном классе, который можно использовать в составе библиотек.
\par
Во введении определены цели и задачи, поставленные в дипломной работе, а также объект и предмет исследования. Целью данной работы является программная реализация функции хэширования и цифровой подписи согласно ГОСТ Р 34.11-2012 и ГОСТ Р 34.10-2012 соответственно.
\par
В заключении сделаны выводы о проделанной работе и подведен итог исследованию. В ходе работы были полученны два программных класса, реализующие функцию хэширования и процедуры генерации и проверки цифровой подписи. С кодом классов можно ознакомиться в приложениях А и В. В дополнение к изложенному, в рамках данной работы был разработан ряд программ для операционной системы Windows с графическим интерфейсом, которые позволяют непосредственно использовать описанные механизмы. С примерами работы данных программ можно ознакомиться в приложениях Б и Г.