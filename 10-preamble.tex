\sloppy

% Настройки стиля ГОСТ 7-32
% Для начала определяем, хотим мы или нет, чтобы рисунки и таблицы нумеровались в пределах раздела, или нам нужна сквозная нумерация.
\EqInChapter % формулы будут нумероваться в пределах раздела
\TableInChapter % таблицы будут нумероваться в пределах раздела
\PicInChapter % рисунки будут нумероваться в пределах раздела

% Добавляем гипертекстовое оглавление в PDF
\usepackage[
bookmarks=true, colorlinks=true, unicode=true,
urlcolor=black,linkcolor=black, anchorcolor=black,
citecolor=black, menucolor=black, filecolor=black,
]{hyperref}

% Изменение начертания шрифта --- после чего выглядит таймсоподобно.
% apt-get install scalable-cyrfonts-tex

\IfFileExists{cyrtimes.sty}
    {
       % \usepackage{cyrtimespatched}
    }
    {
        % А если Times нету, то будет CM...
    }

\usepackage{graphicx}   % Пакет для включения рисунков
\DeclareGraphicsExtensions{.jpg,.pdf,.png}
% С такими оно полями оно работает по-умолчанию:
% \RequirePackage[left=20mm,right=10mm,top=20mm,bottom=20mm,headsep=0pt]{geometry}
% Если вас тошнит от поля в 10мм --- увеличивайте до 20-ти, ну и про переплёт не забывайте:
\geometry{right=20mm}
\geometry{left=30mm}


\usepackage{algorithm}
\usepackage{algpseudocode}

% Перевод плагина
\makeatletter
% Перевод данных об алгоритмах
\renewcommand{\listalgorithmname}{Список алгоритмов}
\floatname{algorithm}{Алгоритм}

%% Перевод команд псевдокода
%\algrenewcommand\algorithmicwhile{\textbf{До тех пока}}
%\algrenewcommand\algorithmicdo{\textbf{выполнять}}
%\algrenewcommand\algorithmicrepeat{\textbf{Повторять}}
%\algrenewcommand\algorithmicuntil{\textbf{Пока выполняется}}
%\algrenewcommand\algorithmicend{\textbf{Конец}}
%\algrenewcommand\algorithmicif{\textbf{Если}}
%\algrenewcommand\algorithmicelse{\textbf{иначе}}
%\algrenewcommand\algorithmicthen{\textbf{тогда}}
%\algrenewcommand\algorithmicfor{\textbf{Цикл}}
%\algrenewcommand\algorithmicforall{\textbf{Выполнить для всех}}
%\algrenewcommand\algorithmicfunction{\textbf{Функция}}
%\algrenewcommand\algorithmicprocedure{\textbf{Процедура}}
%\algrenewcommand\algorithmicloop{\textbf{Зациклить}}
%\algrenewcommand\algorithmicrequire{\textbf{Условия:}}
%\algrenewcommand\algorithmicensure{\textbf{Обеспечивающие условия:}}
%\algrenewcommand\algorithmicreturn{\textbf{Возвратить}}
%\algrenewtext{EndWhile}{\textbf{Конец цикла}}
%\algrenewtext{EndLoop}{\textbf{Конец зацикливания}}
%\algrenewtext{EndFor}{\textbf{Конец цикла}}
%\algrenewtext{EndFunction}{\textbf{Конец функции}}
%\algrenewtext{EndProcedure}{\textbf{Конец процедуры}}
%\algrenewtext{EndIf}{\textbf{Конец условия}}
%\algrenewtext{EndFor}{\textbf{Конец цикла}}
%\algrenewtext{BeginAlgorithm}{\textbf{Начало алгоритма}}
%\algrenewtext{EndAlgorithm}{\textbf{Конец алгоритма}}
%\algrenewtext{BeginBlock}{\textbf{Начало блока. }}
%\algrenewtext{EndBlock}{\textbf{Конец блока}}
%\algrenewtext{ElsIf}{\textbf{иначе если }}
\makeatother

%\usepackage{caption}
%\captionsetup[ruled]{labelsep=period}
%\makeatletter
%\@addtoreset{algorithm}{chapter}% algorithm counter resets every chapter
%\makeatother
%\renewcommand{\thealgorithm}{\thechapter.\arabic{algorithm}}% Algorithm # is %<chapter>.<algorithm>


% Произвольная нумерация списков.
\usepackage{enumerate}
\usepackage{color}

\usepackage{hyperref}
\usepackage{listings}
%\lstset{extendedchars=\true}
\lstset{ %
	language=[Auto]Lisp,
	inputencoding=cp1251,
	basicstyle=\small\sffamily, % размер и начертание шрифта для подсветки кода
	numbers=left,               % где поставить нумерацию строк (слева\справа)
	numberstyle=\tiny,           % размер шрифта для номеров строк
	stepnumber=1,                   % размер шага между двумя номерами строк
	numbersep=5pt,                % как далеко отстоят номера строк от подсвечиваемого кода
	backgroundcolor=\color{white}, % цвет фона подсветки - используем \usepackage{color}
	showspaces=false,            % показывать или нет пробелы специальными отступами
	showstringspaces=false,      % показывать или нет пробелы в строках
	showtabs=false,             % показывать или нет табуляцию в строках
	frame=single,              % рисовать рамку вокруг кода
	tabsize=2,                 % размер табуляции по умолчанию равен 2 пробелам
	captionpos=t,              % позиция заголовка вверху [t] или внизу [b] 
	breaklines=true,           % автоматически переносить строки (да\нет)
	breakatwhitespace=false, % переносить строки только если есть пробел
	escapeinside={-\%*}{*)}   % если нужно добавить комментарии в 
}
%\usepackage{minted}

\setcounter{tocdepth}{2} %Подробность оглавления
%4 это chapter, section, subsection, subsubsection и paragraph
%3 это chapter, section, subsection и subsubsection
%2 это chapter, section, и subsection
%1 это chapter и section
%0 это chapter.
