\Conclusion % заключение к отчёту
\par
Реализация современных криптографических алгоритмов, безусловно, требует высокой квалификации разработчика. Это связанно с тем, что подобные алгоритмы должны работать как можно более эффективно для обеспечения быстродействия обслуживаемых ими систем. Бывает весьма трудоемко оптимальное распределение переменных по регистрам процессора и уровням кэша. Также, особого рассмотрения требует использование встроенных типов. Подобные проблемы часто бывают узкоспециализированны в зависимости от архитектуры ЭВМ и выливаются в широкий спектр прикладных вопросов.
\par
В данной работе были рассмотрены наиболее общие и надежные высокоуровневые языковые конструкции, призванные сформировать у читателя представление о процессах практического вычисления функции хэширования согласно ГОСТ Р 34.11-2012 и реализации процессов генерации и проверки ЭЦП согласно ГОСТ Р 34.10-2012.
\par
В дополнение к изложенному, в рамках данной работы был разработан ряд приложений для операционной системы Windows с графическим интерфейсом, которые позволяют непосредственно использовать описанные механизмы.
\par
Приложения реализуют:
\begin{itemize}
	\item вычисление хэш-функции от файла;
	\item редактирование параметров и ключей схемы цифровой подписи:
	\item генерацию цифровой подписи;
	\item проверку цифровой подписи.
\end{itemize}
