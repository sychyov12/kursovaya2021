\Conclusion % заключение к отчёту
\par
В данной работе была рассмотрена реализация минимизации сожалений для некоторых задач моделирования коррупции. Отработанные в ходе практической реализации модели показывают, что подобные алгоритмы решения могут генерировать устойчивые стратегические профили, которые связаны как с теоритической оценкой, так и с достаточно оптимальным поведением участников. Используемый алгоритм позволяет строить более обобщенные абстракции и проводить анализ используя естественные предположения о информированности игроков относительно всех игровых историй. Основным приемуществом данного подхода являются развитые методики построения решения, которые позволяют в будующем масштабировать результаты для более сложных игровых абстракций. Так, можно	рассматривать произвольную организационную структуру и строить расчеты непосредственно исходя из практических потребностей организации механизмов проверок.
\par
К приемуществам полученной иерархической модели можно отнести масштабируемость с точки зрения проверяемых лиц и свободный выбор альтернатив игроками. Отсутствие случайных событий в базовой постановке позволяет естественным образом отразить целесообразность тех или иных действий. 
\par
К недостаткам можно отнести фиксированный размер штрафов и прочих параметров. В качестве альтернативы можно предложить функциональную зависимость от выплат другим участникам. Также, к возможным доработкам можно отнести и структуру проверяющего органа, который может включать более чем одного игрока, вследствии ограниченных возможностей одного проверяющего агента.
\par 
В целом, полученные алгоритмы и методики могут быть использованы как для изучения старых, так и для построения новых моделей и абстракций. 