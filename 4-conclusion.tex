\Conclusion % заключение к отчёту
\par
В данной работе был продемонстрирован один из лучших на данный момент алгоритмов для приближенного решения игр с неполной информацией. Но несмотря на наличие вполне обобщенных методов, приходится уделять большое внимание разбору частных случаев. Основной проблемой при решении подобных задач является экспоненциальный рост сложности вычислений в зависимости от увеличения числа возможных действий игроков. В связи с этим приходится идти на различные ухищрения с целью получить практически ценный аналог оригинальной задачи. К подобным приемам относят использование метода монте-карло, построение игровых абстракций и многие другие оптимизации.
\par
В качестве объекта исследования была выбрана вполне популярная настольная игра домино. Однако, данной игре уделено довольно мало внимания в контексте рассматриваемого алгоритма. Автор данной работы постарался частично исправить данный недостаток, хотя полученные результаты пока что скромны. Был рассмотрен сильно упрощенный, по сравнению с спортивным, вариант игры с минимумом абстракций. Однако, даже подобный упрощенный вариант раскрывает широкое разнообразие смешанных стратегий, а теоретическая база позволяет говорить о строгой математической обоснованности полученных решений. Для непосредственного расчета стратегий была реализована компьютерная программа.
\par
Дальнейшим развитием данной темы может служить построение более общих абстракций для данной игры. Решения в подобной сфере могут быть полезны как с точки зрения концепции, так и с точки зрения частных методов и оптимизаций.