\chapter{Вторая глава. Программная реализация алгоритма}
\label{cha:ch_2}
\section{Общая схема вычислений}
\section{Первый пример. Покер Куна}

Покер Куна – это максимально упрощенная версия карточной игры покер.

Данная игра достаточно проста, чтобы быть решенной аналитически. Правила следующие:

- в игре участвуют 2 игрока

- игра начинается с раздачи карт игрокам. Всего имеется 3 карты (1, 2 и 3). Каждый игрок получает одну карту. Причем каждый игрок знает свою карту и не знает карту другого игрока.

- по ходу игры игрокам доступны 2 действия «пасс» («п») и «ставка» («с»). Игру начинает первый игрок. Возможны следующие терминальные игровые истории: «пп», «сс», «сп», «псп» и «псс».

- если терминальная игровая история заканчивается на «сс», то игрок с большей картой получает 2 очка, а игрок с меньшей их теряет.

- если терминальная игровая история заканчивается на «сп», то сделавший ставку игрок получает 1 очко, а спасовавший игрок теряет 1 очко.

- если терминальная игровая история заканчивается на «пп», то игроки получают по 0 очков.

Данная игра удобна для базовой проверки алгоритма CFR и часто служит в качестве примера той или иной реализации. Мы можем смоделировать дерево игры и информационные наборы игроков. В данной работе использовался алгоритм MCCFR. При такой реализации все случайные события (раздачи карт) определяются заранее в начале каждой итерации алгоритма. В ходе экспериментов удалось получить следующий профиль стратегий

Основная часть кода программы представлена в приложении А.

\section{Второй пример. Домино}

В данной работе в качестве основного объекта исследования была выбрана игра «Домино».

История

Однако, спортивный вариант игры обладает значительной комбинаторной сложностью и было бы трудно хранить в памяти все дерево игры. В связи с этим в данной работе рассматривались некоторые упрощенные варианты.

Из соображений вычислительной сложности целесообразно рассматривать правила игры следующего вида:

-имеется набор из не более чем 10 костяшек домино;

-игроки имеют на руках 2, 3 (размер руки) костяшки;

-игра может происходить с 2-мя, 3-мя или 4-мя игроками;

-находящиеся не на руках костяшки раздаются по мере развития игры

-игру начинает первый игрок

-все костяшки в процессе игры выкладываются в единственную линию

-если ход возможен, то он происходит по обычным правилам;

-в случае, если ход текущего игрока невозможен, то игра на этом заканчивается, и игроки получают выплаты (победитель забирает все очки, и т.к. необходима нулевая сумма, то проигравшая сторона эти очки теряет).

Приведенные выше правила игры позволяют на практике сформировать полное решение по методу MCCFR. Код приложения для расчета профиля стратегий представлен в приложении Б.

Для проверки полученного приложения был проведен ряд тестовых запусков. Первый тест состоял в определении профиля стратегий для случая игры с полной информацией. Был взят набор из четырех костяшек которые раздавались поровну двум игрокам. Это крайне простая игровая ситуация, но по ней можно судить о работоспособности в целом. Полученный профиль стратегий представлен в приложении В. Для второго теста был выбран набор из шести костяшек домино из которых каждый игрок в начале раунда получал на руки две, а остальные 2 раздавались по ходу игры. Полученный профиль стратегий также представлен в приложении В.

Однако, данные примеры не представляют большой комбинаторной сложности. Для проверки производительности был выбран вариант игры на 10 костяшек для двух игроков. Каждый игрок получал на руки по 3 костяшки и оставшиеся 4 раздавались по ходу игры. Под эксплуатироемостью стратегий подразумевается возможная выгода оппонента, если он будет менять только свою стратегию. Будем под ней понимать максимальный приближенно рассчитанный разброс выигрыша изменившего свою стратегию игрока по сравнению с оригинальным профилем. Назовем эту величину $\tau$. Ниже представлен график расчетной эксплуатируемости стратегий в зависимости от числа обучающих итераций

График