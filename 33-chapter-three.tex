\chapter{Третья глава. Программная реализация алгоритма}
\label{cha:ch_3}
\section{Общая схема вычислений}

\par
В рассмотренных далее примерах рассматривалась вероятностная реализация алгоритма MCCFR. При использовании данного метода значения случайных событий генерируются перед началом каждой обучающей итерации. Данный подход позволяет сократить обьем памяти и ускорить вычисления в некоторых случаях\cite{MCCFR}.
При реализации примеров были выделены следующие компоненты:
\begin{itemize}
	\item описание правил игры (зависит от настроек);
	\item модуль с реализацией алгоритма относительно определенных правил.
\end{itemize}
\par
Настройки игры, например, по возможности могут включать число игроков, состав костяшек домино и т.п.
\par
Правила игры включают структуру игрового дерева, механизм распределения событий и функцию выплат. Игровое дерево строится с применением узлов -- обьектов с информацией о историии игры, о игроке и о возможных действиях.
\par
Сам расчет итераций CFR происходит в выделенном модуле, на основе, определенных для каждого конкретного случая, правил игры. Работа алгоритма начинается с создания игрового дерева. Далее происходит расчет заданного числа итераций. После любой итерации можно получить средние стратегии игроков, которые представляют из себя приближенное коррелирующее равновесие.

\section{Первый пример. Разоблачение совместного преступления}

На рисунке \ref{fig:figkpg} представлен график приближенной эксплуатируемости стратегий. На приведенном графике, и в дальнейшем, горизонтальная ось содержит экспоненциальные отметки о числе итераций по основанию 2. Основная часть кода программы представлена в приложении А.

\section{Второй пример. Иерархическая модель коррупции}

В данной работе в качестве основного объекта исследования была выбрана игра «Иерархическая модель коррупции». 